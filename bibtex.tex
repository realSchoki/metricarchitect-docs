@BOOK{Weske2019,
	AUTHOR = {Weske, Mathias},
	YEAR = {2019},
	TITLE = {Business Process Management - Concepts, Languages, Architectures},
	EDITION = {},
	ISBN = {978-3-662-59432-2},
	PUBLISHER = {Springer},
	ADDRESS = {Berlin, Heidelberg},
}

@BOOK{Dumas2018,
	AUTHOR = {Dumas, Marlon AND Rosa, Marcello La AND Mendling, Jan AND Reijers, Hajo A.},
	YEAR = {2018},
	TITLE = {Fundamentals of Business Process Management - },
	EDITION = {},
	ISBN = {978-3-662-56509-4},
	PUBLISHER = {Springer},
	ADDRESS = {Berlin, Heidelberg},
}

@Inbook{Kirchner2017,
author="Kirchner, Kathrin
and Herzberg, Nico",
editor="Barton, Thomas
and M{\"u}ller, Christian
and Seel, Christian",
title="Ein CMMN-basierter Ansatz f{\"u}r Modellierung und Monitoring flexibler Prozesse am Beispiel von medizinischen Behandlungsabl{\"a}ufen",
bookTitle="Gesch{\"a}ftsprozesse: Von der Modellierung zur Implementierung",
year="2017",
publisher="Springer Fachmedien Wiesbaden",
address="Wiesbaden",
pages="127--145",
abstract="Wissensbasierte Prozesse sind h{\"a}ufig flexibel -- zus{\"a}tzliche Schritte k{\"o}nnen notwendig werden, andere Schritte weggelassen werden, oder die Reihenfolge kann sich {\"a}ndern. Traditionelle Ans{\"a}tze wie BPMN oder UML-Aktivit{\"a}ts-Diagramme haben Schwierigkeiten, diese Flexibilit{\"a}t abzubilden. Zur Unterst{\"u}tzung flexibler Prozesse ver{\"o}ffentlichte die Object Management Group 2014 den neuen Standard Case Management Model {\&} Notation (CMMN). Die Frage, ob CMMN in seiner jetzigen Version in der Praxis eingesetzt werden kann, und wenn ja, wie, ist noch nicht umfassend beantwortet.",
isbn="978-3-658-17297-8",
doi="10.1007/978-3-658-17297-8_7",
url="https://doi.org/10.1007/978-3-658-17297-8_7"
}

@misc{wetterstation,
	keywords = "image"
	AUTHOR = {www.bresser.de},
	TITLE = {BRESSER XXL Wetter-Center JC},
	PUBLISHER = {Springer},
	ADDRESS = {https://www.bresser.de/Wetter-Zeit/Wettercenter/BRESSER-XXL-Wetter-Center-JC-mit-5-in-1-Aussensensor.html},
}
