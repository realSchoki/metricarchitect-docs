% einleitung.tex
\chapter{Introduction}
As the digitization proceeds over time it is important to not get overwhelmed by the complexity growing. \gls{it} assistance is the counterpart guiding and helping society to master the difficulty of this development. To ensure that processes and devices are working and doing what they should, regular measurement and periodic observation are indispensable. In case of failure it provides the necessary information for fixing the problematic situation or even prevent them fully. Build in sensors can be found inside many devices, which can be fundamental for correct functionality. For example engines measuring the rounds per minute, power or other properties. Furthermore common devices like a fridge or an electric oven has thermometers since they need to know how much and when to cool down or heat up. 

The smart home is a topic accessible for everybody without being an \gls{it} professional. It is easy to set up a network of smart home components where room temperature or humidity is tracked, the light can be controlled from the smartphone and the heating is shutting down at night. 

For managing and observing those networks a variety of tools can be used. Most of the smart home sensor devices have proprietary interfaces allowing to only work in a closed system from one manufacturer. Fortunately some devices are able to interconnect with others from various manufacturers. 

\section{Motivation and Background}
Since the complexity rises with the size of the sensor network it is advising to abstract these network and try to create a structure with could help to keep track of it. Not only measuring a handful of devices but every power socket or thermostat in a building and using their values in a reasonable manner is difficult to achieve. It is not only the size of the network but using various different devices and all their varying metrics is another challenge to master. To make non computer scientists able to create such networks a well designed tool is needed that reduces the complexity. Architects or other experts in room design and life quality could then easily create sensor systems without coding.

\section{Structure of the Work}
As the work is based on Prometheus, Grafana and Cinco a brief introduction in each topic will be given. Afterwards the first ideas of an abstraction from a real world scenario will be created. A model and all needed constraints will be developed. Using the results from the previous steps an implementation in Cinco will be done, Furthermore, the model will be modified were it is needed to get the implementation work the right way. Additionally an example will show how working with the generated \gls{ide} looks like. At last a summary will be added, arisen problems will be discussed and final thoughts will be included in an outlook.