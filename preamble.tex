 
% header.tex
\documentclass[a4paper,11pt,twoside,ngerman]{book}
\usepackage[a4paper,left=3.5cm,right=2.5cm,bottom=3.5cm,top=3cm]{geometry}


\usepackage{polyglossia}
\setdefaultlanguage[spelling=new, babelshorthands=true]{german}

\usepackage{fontspec}
\usepackage{unicode-math}
\usepackage{luacode}

%\setromanfont{Cambria}
%\setsansfont{Calibri}
%\setmonofont{Consolas}
\setmonofont{CMU Typewriter Text}
%\setmathfont{Cambria Math}

\usepackage[autostyle]{csquotes}
%\usepackage[]{dirtytalk}
%
\usepackage{graphicx}
\usepackage{color}
\definecolor{LightGray}{RGB}{211, 211, 211}
%
%\usepackage{amsmath,amssymb,subcaption}%,subfigure
\usepackage{subcaption}
\usepackage{float}
\usepackage{needspace}
%
\usepackage{calc}
\usepackage{tikz}
\usetikzlibrary{positioning,calc,arrows,shapes.multipart,arrows.meta}
%
\usepackage[altpo, epsilon]{backnaur}
\renewcommand{\bnfpn}[1]{\textbf{\textrm{#1}}}
\renewcommand{\bnfsp}[0]{,}
\renewcommand{\bnfts}[1]{\textquotedbl\texttt{#1}\textquotedbl}
\newcommand{\bnfop}[1]{[ #1 ]}
\newcommand{\bnfre}[1]{\{ #1 \}}
%%\renewcommand{\bnfpn}[1]{\textbf{\textrm{#1}}}
%%\renewcommand{\bnfts}[1]{\bnf@tsfont{#1}}

%% Theorem-Umgebungen
\usepackage[amsmath,thmmarks]{ntheorem}
%\usepackage{mdframed}
%
%% Algorithmen
\usepackage[plain,chapter]{algorithm}
\usepackage{algorithmic}
%
%% Listings
\usepackage{minted}
%\usepackage{listings}
%
\newmintinline[promcode]{text}{fontsize=\scriptsize, bgcolor=LightGray}
%
%
%\makeatletter
%\let\footnote@orig\footnote
%\def\footnote{%
%	\begingroup
%	\@makeother\#%
%	\footnote@i
%}
%\def\footnote@i#1{%
%	\endgroup
%	\footnote@orig{#1}%
%}
%\makeatother
%
%
%\usepackage{enumerate}
%
%
%% Bibtex deutsch
\usepackage[
backend=biber,
%style=authoryear-icomp,
style=numeric-comp,
sortlocale=de_DE,
natbib=true,
%url=false, 
%doi=true,
%eprint=false
]{biblatex}
\addbibresource{literatur/diplom.bib}
%
\usepackage[]{hyperref}
\hypersetup{
	%colorlinks=true,
	hidelinks
}
%
%% URLs
%\usepackage{url}
%
%% Abbr
\usepackage{glossaries}
\makeindex

\newglossaryentry{latex}
{
        name=latex,
        description={Is a mark up language specially suited for 
scientific documents}
}

\newglossaryentry{maths}
{
        name=mathematics,
        description={Mathematics is what mathematicians do}
}

\newglossaryentry{formula}
{
        name=formula,
        description={A mathematical expression}
}

\newacronym{iot}{IoT}{Internet of things}
\newacronym{pql}{PQL}{Prometheus Query Language}
\newacronym{tsdb}{TSDB}{Time Series Database}
\newacronym{promql}{PromQL}{Prometheus Query Language}
\newacronym{bzgl}{bzgl.}{bezüglich}
\newacronym{IP}{IP}{Internet Protocol}
\newacronym{IPv4}{IPv4}{Internet Protocol version 4}
\newacronym{IPv6}{IPv6}{Internet Protocol version 6}
\newacronym{ip-address}{IP-address}{Internet Protocol Address}
\newacronym{http}{HTTP}{Hyper Text Transfer Protocol}
\newacronym{ebnf}{EBNF}{Extended Backus–Naur form}
\newacronym{json}{JSON}{JavaScript Object Notation}
\newacronym{tudortmund}{TU Dortmund}{Technische Universität Dortmund}
\newacronym{ide}{IDE}{integrated development environment}
\newacronym{dsl}{DSL}{domain specific language}
\newacronym{mgl}{MGL}{meta graph language}
\newacronym{msl}{MSL}{meta style language}
\newacronym{w}{W}{Watt}
\newacronym{percent}{\%}{percent}
\newacronym{degreeC}{$^\circ$C}{degree celcius}
\newacronym{url}{URL}{Uniform Resource Locator}
\newacronym{ui}{UI}{User Interface}
\newacronym{emf}{EMF}{Eclipse Modeling Framework}
\newacronym{uml}{UML}{Unified Modeling Language}
%
%% Caption Packet
%\usepackage[margin=0pt,font=small,labelfont=bf]{caption}
%% Gliederung einstellen
%%\setcounter{secnumdepth}{5}
%%\setcounter{tocdepth}{5}

% Theorem-Optionen %
%\theoremseparator{.}
%\theoremstyle{change}
\newtheorem{Definition}{Definition}
\newtheorem{Beispiel}{Beispiel}
%\newtheorem{theorem}{Theorem}[section]
%\newtheorem{satz}[theorem]{Satz}
%\newtheorem{lemma}[theorem]{Lemma}
%\newtheorem{korollar}[theorem]{Korollar}
%\newtheorem{proposition}[theorem]{Proposition}
% Ohne Numerierung
%\theoremstyle{nonumberplain}
%\renewtheorem{theorem*}{Theorem}
%\renewtheorem{satz*}{Satz}
%\renewtheorem{lemma*}{Lemma}
%\renewtheorem{korollar*}{Korollar}
%\renewtheorem{proposition*}{Proposition}
%% Definitionen mit \upshape
%\theorembodyfont{\upshape}
%\newtheorem{definition}[theorem]{Definition}
%%\theoremstyle{nonumberplain}
%\renewtheorem{definition*}{Definition}
%% Kursive Schrift
\theoremheaderfont{\itshape}
%\newtheorem{notation}{Notation}
%\newtheorem{konvention}{Konvention}
%\newtheorem{bezeichnung}{Bezeichnung}
%\newtheorem{beweis}{Beweis}
%\theoremsymbol{}
%\theoremheaderfont{\bfseries}
%\newtheorem{bemerkung}[theorem]{Bemerkung}
%\newtheorem{beobachtung}[theorem]{Beobachtung}
%\newtheorem{beispiel}[theorem]{Beispiel}
%\newtheorem{problem}{Problem}
%%\theoremstyle{nonumberplain}
%\renewtheorem{bemerkung*}{Bemerkung}
%\renewtheorem{beispiel*}{Beispiel}
%\renewtheorem{problem*}{Problem}

%% Algorithmen anpassen %
%\renewcommand{\algorithmicrequire}{\textit{Eingabe:}}
%\renewcommand{\algorithmicensure}{\textit{Ausgabe:}}
%\floatname{algorithm}{Algorithmus}
%\renewcommand{\listalgorithmname}{Algorithmenverzeichnis}
%\renewcommand{\algorithmiccomment}[1]{\color{grau}{// #1}}

% Zeilenabstand einstellen %
\renewcommand{\baselinestretch}{1.25}
% Floating-Umgebungen anpassen %
\renewcommand{\topfraction}{0.9}
\renewcommand{\bottomfraction}{0.8}
% Abkuerzungen richtig formatieren %
\usepackage{xspace}
\newcommand{\vgl}{vgl.\@\xspace} 
\newcommand{\zB}{z.\nolinebreak[4]\hspace{0.125em}\nolinebreak[4]B.\@\xspace}
\newcommand{\bzw}{bzw.\@\xspace}
\newcommand{\dahe}{d.\nolinebreak[4]\hspace{0.125em}h.\nolinebreak[4]\@\xspace}
\newcommand{\etc}{etc.\@\xspace}
\newcommand{\evtl}{evtl.\@\xspace}
\newcommand{\ggf}{ggf.\@\xspace}
\newcommand{\bzgl}{bzgl.\@\xspace}
\newcommand{\so}{s.\nolinebreak[4]\hspace{0.125em}\nolinebreak[4]o.\@\xspace}
\newcommand{\iA}{i.\nolinebreak[4]\hspace{0.125em}\nolinebreak[4]A.\@\xspace}
\newcommand{\sa}{s.\nolinebreak[4]\hspace{0.125em}\nolinebreak[4]a.\@\xspace}
\newcommand{\su}{s.\nolinebreak[4]\hspace{0.125em}\nolinebreak[4]u.\@\xspace}
\newcommand{\ua}{u.\nolinebreak[4]\hspace{0.125em}\nolinebreak[4]a.\@\xspace}
\newcommand{\og}{o.\nolinebreak[4]\hspace{0.125em}\nolinebreak[4]g.\@\xspace}
\newcommand{\oBdA}{o.\nolinebreak[4]\hspace{0.125em}\nolinebreak[4]B.\nolinebreak[4]\hspace{0.125em}d.\nolinebreak[4]\hspace{0.125em}A.\@\xspace}
\newcommand{\OBdA}{O.\nolinebreak[4]\hspace{0.125em}\nolinebreak[4]B.\nolinebreak[4]\hspace{0.125em}d.\nolinebreak[4]\hspace{0.125em}A.\@\xspace}

% Leere Seite ohne Seitennummer, naechste Seite rechts
\newcommand{\blankpage}{
	\clearpage{\pagestyle{empty}\cleardoublepage}
}

% Keine einzelnen Zeilen beim Anfang eines Abschnitts (Schusterjungen)
\clubpenalty = 10000
% Keine einzelnen Zeilen am Ende eines Abschnitts (Hurenkinder)
\widowpenalty = 10000 \displaywidowpenalty = 10000
% EOF